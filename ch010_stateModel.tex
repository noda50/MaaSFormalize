%%======================================================================
\chapter{利用申告とその価値}
%\section{利用申告とその価値}
\label{ch:利用申告とその価値}
%% = = = = = = = = = = = = = = = = = = = = = = = = = = = = = = = = = = =

%%----------------------------------------------------------------------
\section{利用申告モデル}
\label{s:利用申告モデル}
%% - - - - - - - - - - - - - - - - - - - - - - - - - - - - - - - - - - -

%%--------------------------------------------------
\subsection{利用申告}
%% - - - - - - - - - - - - - - - - - - - - - - - - -
利用申告\footnote{statement} $s_{a}(\V{d})$ を以下のように定義する。
  %%$$$$$$$$$$$$$$$$$$$$$$$$$$$$$$$$$$$$$$$$
  \begin{eqnarray}
    \textrm{利用申告} s_a(\V{d}) & : & \textrm{実デマンド $\V{d}$ を発生する確率}
  \\
    \textrm{デマンド} \V{d} & = & \Tuple{t,o,d,n}
  \\
    a & : & \textrm{利用者}
  \\
    t & : & \textrm{利用時刻}
  \\
    o,d & : & \textrm{乗車場所,降車場所}
  \\
    n & : & \textrm{乗車人数}
  \end{eqnarray}
  %%$$$$$$$$$$$$$$$$$$$$$$$$$$$$$$$$$$$$$$$$

%%--------------------------------------------------
\subsection{申告集合}
%% - - - - - - - - - - - - - - - - - - - - - - - - -
未申告者を含む利用者全員の利用申告の集合(申告集合)を以下のように定義する。
  %%$$$$$$$$$$$$$$$$$$$$$$$$$$$$$$$$$$$$$$$$
  \begin{eqnarray}
    \textrm{申告集合} \V{S} & = & \Set{s_a | a \in \S{A}}
  \\
    \V{S}(\V{D}) & : & \textrm{デマンド集合$\V{D}$が発生する確率}
  \\
    \V{D} & : & \Set{\V{d}_a | a \in \S{A}} : \textrm{全員分のデマンド}
  \end{eqnarray}
  %%$$$$$$$$$$$$$$$$$$$$$$$$$$$$$$$$$$$$$$$$
ただし、未申告は、情報量(曖昧さ)最大となる分布としておく。

%%--------------------------------------------------
\subsection{運行計画}
%% - - - - - - - - - - - - - - - - - - - - - - - - -
運行計画 $\pi$ は、運行台数の配分、配車アルゴリズムなどの組であるとする。

可能な運行計画の集合を$\S{\Pi}$で表すものとする。

%%--------------------------------------------------
\subsection{実運行およびシミュレーション}
%% - - - - - - - - - - - - - - - - - - - - - - - - -
あるデマンド集合 $\V{D}$ と運行計画 $\pi$ が与えられている時、
実際の運行(実運行)を関数 $\V{f}$ で表すものとする。
  %%$$$$$$$$$$$$$$$$$$$$$$$$$$$$$$$$$$$$$$$$
  \begin{eqnarray}
    \V{f}(\V{D},\pi) & = & \textrm{運行した結果の経緯を表すデータ}
  \end{eqnarray}
  %%$$$$$$$$$$$$$$$$$$$$$$$$$$$$$$$$$$$$$$$$
また、運行状態を推定するシミュレーション関数を $\Est{\V{f}}$ と表しておく。
  %%$$$$$$$$$$$$$$$$$$$$$$$$$$$$$$$$$$$$$$$$
  \begin{eqnarray}
    \Est{\V{f}}(\V{D},\pi) & = & \textrm{運行シミュレーションの出力}
  \end{eqnarray}
  %%$$$$$$$$$$$$$$$$$$$$$$$$$$$$$$$$$$$$$$$$
なお、実運行、運行シミュレーションは共に確率的過程を含むものとし、
同じ引数でも同じ結果とならないものと仮定する。
(数式的にどう標記すべき?)
  
%%--------------------------------------------------
\subsection{運行評価関数}
%% - - - - - - - - - - - - - - - - - - - - - - - - -
実デマンド $\V{D}$ に対する運行結果の評価関数を $u$ とする。
  %%$$$$$$$$$$$$$$$$$$$$$$$$$$$$$$$$$$$$$$$$
  \begin{eqnarray}
    u(\V{f}(\V{D}, \pi) & : &
    \textrm{デマンド集合 $\V{D}$ を計画 $\pi$ で運行した時の
            性能を表すデータ}
  \end{eqnarray}
  %%$$$$$$$$$$$$$$$$$$$$$$$$$$$$$$$$$$$$$$$$
ただし、$u$ の出力はスカラー値・ベクトル両方あり得るとする。

%%--------------------------------------------------
\subsection{予測評価関数}
%% - - - - - - - - - - - - - - - - - - - - - - - - -
運行評価関数を使って、申告集合 $\V{S}$ に対する予測評価関数は、
以下のように定義する。
  %%$$$$$$$$$$$$$$$$$$$$$$$$$$$$$$$$$$$$$$$$
  \begin{eqnarray}
    \Est{U}(\pi | \V{S})
      & = &
        \oint u(f(\V{D},\pi)) \V{S}(\V{D}) d \V{D}
  \end{eqnarray}
  %%$$$$$$$$$$$$$$$$$$$$$$$$$$$$$$$$$$$$$$$$

%%--------------------------------------------------
\subsection{最適運行計画}
%% - - - - - - - - - - - - - - - - - - - - - - - - -
予測評価関数を用いて、最適運行計画を以下のように定義する。
  %%$$$$$$$$$$$$$$$$$$$$$$$$$$$$$$$$$$$$$$$$
  \begin{eqnarray}
    \Opt{\pi}(\V{S})
      & = &
        \arg\max_{\pi \in \S{\Pi}} \Est{U}(\pi | \V{S})
  \end{eqnarray}
  %%$$$$$$$$$$$$$$$$$$$$$$$$$$$$$$$$$$$$$$$$

%%--------------------------------------------------
\subsection{申告更新の価値}
%% - - - - - - - - - - - - - - - - - - - - - - - - -
ある利用者エージェント $a$ 申告更新 $s_a \to s'_a$ とは、
$a$ の利用申告を、$s_a$ から $s'_a$ に変更することとする。
この申告更新を含んだ申告集合の変化は、以下のようになる。
  %%$$$$$$$$$$$$$$$$$$$$$$$$$$$$$$$$$$$$$$$$
  \begin{eqnarray}
    \V{S}' & = & (\V{S} \backslash \Set{s_a}) \cup\Set{s'_a}
  \end{eqnarray}
  %%$$$$$$$$$$$$$$$$$$$$$$$$$$$$$$$$$$$$$$$$

これらに基づき、
申告更新 $s_a \to s'_a$ の価値 $v$ は、
以下のように表す。
  %%$$$$$$$$$$$$$$$$$$$$$$$$$$$$$$$$$$$$$$$$
  \begin{eqnarray}
    v(s_a \to s'_a) & = & v(\V{S} \to \V{S}')
  \\
      & = &
        \Est{U}(\Opt{\pi}' | \V{S}') -
        \Est{U}(\Opt{\pi} | \V{S})
  \\
    \Opt{\pi}' & = & \Opt{\pi}(\V{S}')
  \\
    \Opt{\pi} & = & \Opt{\pi}(\V{S})
  \end{eqnarray}
  %%$$$$$$$$$$$$$$$$$$$$$$$$$$$$$$$$$$$$$$$$

%%--------------------------------------------------
\subsection{最大有効申告更新(者)}
%% - - - - - - - - - - - - - - - - - - - - - - - - -

  %%$$$$$$$$$$$$$$$$$$$$$$$$$$$$$$$$$$$$$$$$
  \begin{eqnarray}
    \arg\max_{s'_a} v(s_a \to s'_a)
    \\
    \arg\max_{a} v(s_a \to s'_a)
  \end{eqnarray}
  %%$$$$$$$$$$$$$$$$$$$$$$$$$$$$$$$$$$$$$$$$


