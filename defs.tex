%%----------------------------------------------------------------------
%%
\usepackage{amssymb}
\usepackage{bm}
%\def\V#1{\mbox{\boldmath $#1$}}
%\def\M#1{\mbox{\boldmath $#1$}}
%\def\S#1{\mbox{\boldmath $#1$}}
%\def\Ts#1{\mbox{\boldmath $#1$}}
\def\V#1{{\bm{#1}}}
\def\M#1{{\bm{#1} }}
\def\S#1{{\bm{#1}}}
\def\Ts#1{{\bm{#1}}}

\def\MtxK#1#2{\left[ \begin{array}[c]{#1} #2 \end{array} \right]}
\def\Vec#1{\MtxK{c}{#1}}
\def\Mtx2#1{\MtxK{cc}{#1}}
\def\Mtx1#1{\MtxK{c}{#1}}
\def\Tsr#1{\MtxK{c}{#1}}

\def\DpO{\times} %% 直積記号 : direct product
\def\DpA#1{\bar{#1}}  %% 直積の値の頭につけるアクセント

\def\MxA#1{\hat{#1}}  %%  混合分布につけるアクセント

\usepackage{accents}
\def\Est#1{\hat{#1}}    %% estimate
\def\Apx#1{\tilde{#1}}  %% approx
%\def\Opt#1{\check{#1}}  %% optimum
\def\Opt#1{\accentset{*}{#1}}  %% optimum

\def\Av#1{\left|#1\right|}
\def\SD#1{\sigma_{#1}}
\def\Var#1{\SD{#1}^2}

\def\Seq#1{\left\{ #1 \right\}}
\def\Set#1{\left\{ #1 \right\}}
\def\Tuple#1{\left< #1 \right>}

\makeatletter
  \def\@LkS{{\cal L}}    %% likelihood Symbol
  \def\@LkF[#1]{\@LkS\left(#1\right)}    %% likelihood func.
  \def\Lk{\@ifnextchar[{\@LkF}{\@LkS}% ]
         }
\makeatother

\def\Pr{{\cal P}}
\def\G{{\cal G}}
\def\nz{\epsilon}	%% noise

%%======================================================================
\def\NofI{i}  %% 罹患者数
\def\NofD{v}  %% 発症者数
\def\CumD{w}  %% 累計発症者数

\def\LpNofI{I}  %% 罹患者数(ラプラス変換後)
\def\LpNofD{V}  %% 発症者数(ラプラス変換後)
\def\LpCumD{W}  %% 累計発症者数(ラプラス変換後)

\def\PrR{\gamma} %% 治癒確率
\def\PrD{\lambda} %% 発症確率

\def\Ddt#1{\dot{#1}} %% 時間微分

\makeatletter
\def\Lp{\@ifnextchar[{\@Lp}{{\cal L}}} %% ラプラス変換 \Lp[f] or \Lp(f)
\def\@Lp[#1]{\Lp{}\left[#1\right]}
\def\ILp{\@ifnextchar[{\@ILp}{{\cal L}^{-1}}} %% ラプラス変換 \Lp[f] or \Lp(f)
\def\@ILp[#1]{\ILp{}\left[#1\right]}
\makeatother

\def\Conv{\ast}  %% 畳み込み積分 (Convolution Integral)


%%======================================================================

\def\Where{\mbox{\hspace{3em}} ; \mbox{\hspace{1em}}}
\def\where0pt{\makebox[0pt][l]{\mbox{where}} \nonumber}


\def\MyBox#1{{\unitlength=#1 \framebox(1,1){}}}
\def\MyFillBox#1{\rule{#1}{#1}}
\def\EOTH{\hfill\MyBox{.5em}}
\def\QED{\hfill\MyFillBox{.5em}}

\newenvironment{Note}[1]{%
%  \paragraph*{ノート: #1} \fill \rule{0.5\textwidth}{1pt}
  \paragraph{\protect\MyFillBox{1em} ノート--- #1} \hrule%
  \hfill 
}{%
  \hfill\MyBox{1em}\hrule
  \ \\
}

%%======================================================================

\newif\ifWideBoxedEqnarray
\WideBoxedEqnarraytrue
\WideBoxedEqnarrayfalse

\newenvironment{BoxedEqnarray}{%
  \vspace{-4ex}
  \begin{center}%
    \begin{tabular}[c]{|c|} \hline%
      \begin{minipage}[c]{\textwidth}  %{width}
      \ifWideBoxedEqnarray%
        \vspace{-0.5ex} %
      \else%
        \vspace{-2.3ex} %
      \fi%
        \begin{eqnarray}%
}{%
        \end{eqnarray}%
      \end{minipage}%
      \ifWideBoxedEqnarray%
        \\%
      \else%
        % nothing
      \fi%
%      \\
      \\ \hline%
    \end{tabular}
  \end{center}%
}

%%======================================================================

\def\DefPrefix{定義}
\def\LemmaPrefix{補題}
\def\TheoremPrefix{定理}
\newtheorem{genericTheorems}{GenericTheorem}[section]
\newtheorem{xDef}[genericTheorems]{\DefPrefix}
\newtheorem{xLemma}[genericTheorems]{\LemmaPrefix}
\newtheorem{xTheorem}[genericTheorems]{\TheoremPrefix}

\newenvironment{Lemma}{%
  \begin{xLemma}%
    \addcontentsline{toc}{subparagraph}{%
      \fbox{\LemmaPrefix \thexLemma}}
  \ \\
}{%
  \EOTH%
  \end{xLemma}%
}

\newenvironment{Theorem}{%
  \begin{xTheorem}%
    \addcontentsline{toc}{subparagraph}{%
      \fbox{\TheoremPrefix \thexLemma}}
  \ \\
}{%
  \EOTH
  \end{xTheorem}%
}



